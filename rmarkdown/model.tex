\documentclass[]{article}
\usepackage{lmodern}
\usepackage{amssymb,amsmath}
\usepackage{ifxetex,ifluatex}
\usepackage{fixltx2e} % provides \textsubscript
\ifnum 0\ifxetex 1\fi\ifluatex 1\fi=0 % if pdftex
  \usepackage[T1]{fontenc}
  \usepackage[utf8]{inputenc}
\else % if luatex or xelatex
  \ifxetex
    \usepackage{mathspec}
  \else
    \usepackage{fontspec}
  \fi
  \defaultfontfeatures{Ligatures=TeX,Scale=MatchLowercase}
\fi
% use upquote if available, for straight quotes in verbatim environments
\IfFileExists{upquote.sty}{\usepackage{upquote}}{}
% use microtype if available
\IfFileExists{microtype.sty}{%
\usepackage{microtype}
\UseMicrotypeSet[protrusion]{basicmath} % disable protrusion for tt fonts
}{}
\usepackage[margin=1in]{geometry}
\usepackage{hyperref}
\hypersetup{unicode=true,
            pdftitle={Model of translational coupling},
            pdfborder={0 0 0},
            breaklinks=true}
\urlstyle{same}  % don't use monospace font for urls
\usepackage{graphicx,grffile}
\makeatletter
\def\maxwidth{\ifdim\Gin@nat@width>\linewidth\linewidth\else\Gin@nat@width\fi}
\def\maxheight{\ifdim\Gin@nat@height>\textheight\textheight\else\Gin@nat@height\fi}
\makeatother
% Scale images if necessary, so that they will not overflow the page
% margins by default, and it is still possible to overwrite the defaults
% using explicit options in \includegraphics[width, height, ...]{}
\setkeys{Gin}{width=\maxwidth,height=\maxheight,keepaspectratio}
\IfFileExists{parskip.sty}{%
\usepackage{parskip}
}{% else
\setlength{\parindent}{0pt}
\setlength{\parskip}{6pt plus 2pt minus 1pt}
}
\setlength{\emergencystretch}{3em}  % prevent overfull lines
\providecommand{\tightlist}{%
  \setlength{\itemsep}{0pt}\setlength{\parskip}{0pt}}
\setcounter{secnumdepth}{0}
% Redefines (sub)paragraphs to behave more like sections
\ifx\paragraph\undefined\else
\let\oldparagraph\paragraph
\renewcommand{\paragraph}[1]{\oldparagraph{#1}\mbox{}}
\fi
\ifx\subparagraph\undefined\else
\let\oldsubparagraph\subparagraph
\renewcommand{\subparagraph}[1]{\oldsubparagraph{#1}\mbox{}}
\fi

%%% Use protect on footnotes to avoid problems with footnotes in titles
\let\rmarkdownfootnote\footnote%
\def\footnote{\protect\rmarkdownfootnote}

%%% Change title format to be more compact
\usepackage{titling}

% Create subtitle command for use in maketitle
\newcommand{\subtitle}[1]{
  \posttitle{
    \begin{center}\large#1\end{center}
    }
}

\setlength{\droptitle}{-2em}
  \title{Model of translational coupling}
  \pretitle{\vspace{\droptitle}\centering\huge}
  \posttitle{\par}
  \author{}
  \preauthor{}\postauthor{}
  \date{}
  \predate{}\postdate{}


\begin{document}
\maketitle

\subsection{A model of translational
coupling}\label{a-model-of-translational-coupling}

To model the effects of translational coupling on protein production, we
first assume a polycistronic transcript with three genes \emph{a},
\emph{b}, and \emph{c}. For the effective initiation rate of
\(a_\text{in}\) we define the following relationship
\[a_\text{in} = \min\left\{i_a, \tau_a\right\}\] where \(i_a\) is the
aggregate initiation rate of \emph{a} and \(\tau_a\) is the translation
elongation rate of gene \emph{a}. We assume that if the initiation rate
ever exceeds the elongation rate, ribosomes would quickly back up on the
transcript and make elongation the rate-limiting step of translation.
Thus in our model, the elongation rate can never be exceed by the
aggregate translation initiation rate. For gene \emph{a}, the aggregate
initiation rate is simply the \emph{de novo} initiation rate because
there are no genes upstream of \emph{a}.

The rate at which ribosomes complete translation of gene \(a\), or
protein production rate of A, is defined as

\[ \dot{A} = a_\text{in} \] at steady state. In this context, the steady
state assumption means that all three protein products are being formed
continuously.

For the effective translation initiation rate \(b_\text{in}\) of gene
\emph{b}, we similarly write
\[b_\text{in} = \min\left\{i_b, \tau_b\right\}\]

where \(i_b\) is the aggregate translation initiation rate due to
upstream-dependent reinitiation and \textit{de novo} initiation, and
\(\tau_b\) is the translation elongation rate of \emph{b}. We define the
aggregate translation initiation rate
\[i_b = b_\text{reinit} + b_\text{de novo}\] where \(b_\text{reinit}\)
is the rate of upstream translating ribosomes reinitiating on gene
\textit{b} and \(b_\text{de novo}\) is the rate of ribosomes initiating
\emph{de novo} on gene \emph{b}.

Lastly we define reinitiation and \emph{de novo} initiation rates on
gene \emph{b} as follows

\[b_\text{reinit} = q_b\dot{A}\] \[b_\text{de novo} = z_b\dot{A} + w_b\]
where \(\dot{A}\) is the rate of ribosomes flowing from upstream
translation of gene \(a\), and \(q_b\) represents the proportion of that
ribosome flow reinitiating on gene \emph{b}. We assume that the rate of
\emph{de novo} initation depends, in part, on upstream ribosomes
relaxing secondary structure around the ribosome binding site of gene
\emph{b}. Thus, the rate of \emph{de novo} initiation is given by the
upstream ribosome flow \(\dot{A}\) scaled by some constant \(z_b\), and
by a constant rate \(w_b\) that does not depend on upstream ribosome
flow.

We can simplify the effective initiation rate to
\[b_\text{in} = \min\left\{\dot{A}(q_b + z_b) + w_b, \tau_b\right\}\]

We simplify the effective initiation rate further by defining a coupling
constant \[
y_b = q_b + z_b
\] which accounts for both the effects of facilitated binding and
ribosome reinitiation. The final effective initiation rate of \emph{b}
is defined as

\[
b_\text{in} = \min\left\{y_b\dot{A} + w_b, \tau_b\right\}
\]

Similar to the protein production rate of gene \emph{a}, we define the
protein production of gene \emph{b}

\[ \dot{B} = b_\text{in} \]

at steady state.

The effective initiation rate of gene \emph{c} is similar to that of
gene \emph{b}

\[
c_\text{in} = \min\left\{y_c\dot{B} + w_c, \tau_c\right\}
\] \[
y_c = q_c + z_c
\]

where \(\dot{B}\) is the rate of ribosomes flowing from the end of
upstream gene \(b\), and \(q_c\) represents the proportion of that
ribosome flow reinitiating on gene \emph{c}. The rate of \emph{de novo}
initiation dependent on upstream ribosome flow is given by \(z_c\), and
\(w_c\) is the \emph{de novo} initation rate independent of upstream
ribosome flow. Again, \(y_c\) is a coupling constant that incorporates
the effects of both facilitated binding and ribosome reinitiation.

\subsubsection{Results and figures}\label{results-and-figures}

I have made plots similar to Fig 7 in our paper over a range of \(y\)
and \(w\) values at steady state. We assume that \(y_c = y_b\) and
\(w_b = w_c\). We also assume that \(\tau_a = \tau_c\), and we vary
\(\tau_b\).

\includegraphics{model_files/figure-latex/model_fig-1.pdf}


\end{document}
